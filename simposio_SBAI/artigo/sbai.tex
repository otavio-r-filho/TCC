\documentclass[conference,harvard,brazil,english]{sbatex}
\usepackage[latin1]{inputenc}
\usepackage[T1]{fontenc}
\usepackage{lipsum}
\usepackage{ae}
%
% LaTeX2e class SBATeX
%
% Vers�o 1.0 alpha
%   Walter Fetter Lages
%   w.fetter@ieee.org
%
% Este arquivo sbai2013.tex � uma adapta��o do arquivo revista.tex,
% Vers�o: 1.0 alpha, desenvolvido por Maur�cio C. de Oliveira,
% mcdeoliveira@ieee.org.
%
% As adapta��es fazem com que, por default, sejam utilizadas
% as op��es adequadas para o formato do SBAI 2013, ao contr�rio do arquivo
% revista.tex, que, por default, utiliza op��es adequadas para o formato
% da Revista da SBA.
%
%
% --------------------------------------------------
%
% Para compilar este exemplo use a seq��ncia de comandos:
%
%     latex sbai2013
%     bibtex sbai2013
%     latex sbai2013
%     latex sbai2013
%
% Para gerar um arquivo Postscript (.ps):
%
%     dvips -t a4 sbai2013
%
% Para gerar um arquivo Portable Document Format (.pdf):
%
%     dvips -Ppdf -t a4 sbai2013
%     ps2pdf -dMaxSubsetPct=100 -dSubsetFonts=true -dEmbedAllFonts=true -dCompatibilityLevel=1.2 -sPAPERSIZE=a4 sbai2013.ps
%
% --------------------------------------------------
%  Este arquivo foi testado no:
%   - Windows: Miktex 2.9 + TeXnicCenter 1.0 Stable Release
%   - Linux  : Tex Live 2007.dfsg.2-7ubuntu1.1 + Kile 2.0.83
%              Tex Live 2007.dfsg.2-7ubuntu1.1 + Emacs 22 (GTK)
% --------------------------------------------------
%
% --------------------------------------------------
%  Estes comandos s�o necess�rios apenas para a
%  a gera��o deste artigo exemplo. Eles n�o fazem
%  parte do estilo SBATeX.
% --------------------------------------------------
\makeatletter
\def\verbatim@font{\normalfont\ttfamily\footnotesize}
\makeatother
\usepackage{amsmath}
% --------------------------------------------------


\begin{document}

% CABE�ALHO

\title{Mapeamento 2D por Sensoriamento � Laser}

\author{Ot�vio Vicente Ribeiro}{otavio.r.filho@gmail.com}
\address{UFBA - Escola Polit�cnica - DEE\\
	     40210-630 Salvador, BA}

% Com a op��o 'journal' pode-se definir
%\volume{X}
%\numero{X}
%\mes{Jan e Fev}
%\ano{2010}
% Caso contr�rio voc� ver� um irritante aviso de
% que estes valores n�o foram definidos.

% \twocolumn apenas para conference
\twocolumn[

\maketitle

\selectlanguage{english}
\begin{abstract}
  This is where the abstract should be placed. It is a single paragraph providing a concise summary of the material in the paper below. It usually has less than 200 words. It should not be confused with the introduction, must not contain abbreviations, footnotes, references to literature, figures, etc.   \end{abstract}

\keywords{Template, Example.}

\selectlanguage{brazil}
\begin{abstract}
  Os artigos a serem submetidos dever�o ser redigidos em l�ngua portuguesa, espanhola ou inglesa, com n�mero m�ximo de 6 (seis) p�ginas, tamanho A4, coluna dupla, em formato PDF.
\end{abstract}

\keywords{Lista de palavras-chave, separadas por v�rgulas. M�ximo de quatro palavras-chave, sendo que pelo menos uma delas deve corresponder a um dos t�picos de interesse dos eventos. Conferir t�picos em: http://www.sbai2013.ufc.br/ ou http://posmec.ufabc.edu.br/dincon2013/).}
]

% CONTRIBUI��O

\selectlanguage{brazil}

\section{Introdu��o}
Inserir texto de introdu��o\\
\lipsum

\input{sections/ambiente_simulacao}

\section*{Agradecimentos}
Mencione aqui seus agradecimentos as ag�ncias de fomento e colaboradores no trabalho.

% BIBLIOGRAFIA
\bibliography{exemplo}
\end{document}
